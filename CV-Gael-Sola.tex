%%%%%%%%%%%%%%%%%%%%%%%%%%%%%%%%%%%%%%%%%
% "ModernCV" CV and Cover Letter
% LaTeX Template
% Version 1.3 (29/10/16)
%
% This template has been downloaded from:
% http://www.LaTeXTemplates.com
%
% Original author:
% Xavier Danaux (xdanaux@gmail.com) with modifications by:
% Vel (vel@latextemplates.com)
%
% License:
% CC BY-NC-SA 3.0 (http://creativecommons.org/licenses/by-nc-sa/3.0/)
%
% Important note:
% This template requires the moderncv.cls and .sty files to be in the same 
% directory as this .tex file. These files provide the resume style and themes 
% used for structuring the document.
%
%%%%%%%%%%%%%%%%%%%%%%%%%%%%%%%%%%%%%%%%%

%----------------------------------------------------------------------------------------
%	PACKAGES AND OTHER DOCUMENT CONFIGURATIONS
%----------------------------------------------------------------------------------------

\documentclass[11pt,a4paper,sans]{moderncv} % Font sizes: 10, 11, or 12; paper sizes: a4paper, letterpaper, a5paper, legalpaper, executivepaper or landscape; font families: sans or roman

%------ Set font for the whole doc

%\usepackage{DejaVuSans}
%\usepackage{tgadventor} %needs tlmgr_install("tex-gyre")
%\usepackage{lmodern}
%\usepackage[scaled=1.2]{libertine}
%\usepackage{lato}
\usepackage{fontawesome}
%\usepackage{palatino} %tlmgr_install("palatino")
%\usepackage{charter} %tlmgr_install("charter")
\renewcommand*\familydefault{\sfdefault} %% Only if the base font of the document is to be sans serif
\usepackage[T1]{fontenc}

%------ Set specific font for name and section
\newenvironment{myfont}{\fontfamily{bch}\selectfont}{\par} % select charter as special font
\DeclareTextFontCommand{\textmyfont}{\myfont} % replace text inside \textmyfont{} to charter

%\renewcommand{\sfdefault}{bch}

%------ Set theme

\moderncvstyle{classic} % CV theme - options include: 'casual' (default), 'classic', 'oldstyle' and 'banking'
\moderncvcolor{green} % CV color - options include: 'blue' (default), 'orange', 'green', 'red', 'purple', 'grey' and 'black'

\usepackage{lipsum} % Used for inserting dummy 'Lorem ipsum' text into the template

%\usepackage[scale=0.75]{geometry} % Reduce document margins
\usepackage[scale=0.8, top=2cm, bottom=2cm, left=2cm, right=2cm]{geometry}
%\setlength{\hintscolumnwidth}{3cm} % Uncomment to change the width of the dates column
%\setlength{\makecvtitlenamewidth}{10cm} % For the 'classic' style, uncomment to adjust the width of the space allocated to your name

%\usepackage[utf8]{inputenc}   

%%% autorise tables
\setlength\arrayrulewidth{.4pt}
\setlength\tabcolsep{6pt}

%%% tabular ajusté à la page
\usepackage{array}
  \newcolumntype{x}[1]{>{\hspace{0pt}}p{#1}}
	\newcolumntype{R}[1]{>{\raggedleft\hspace{0pt}}p{#1}}

\usepackage{booktabs}  
\usepackage{tabularx}                                         
  \setlength{\doublerulesep}{\arrayrulewidth}  

\usepackage{natbib}

%----------------------------------------------------------------------------------------
%	NAME AND CONTACT INFORMATION SECTION
%----------------------------------------------------------------------------------------

\firstname{Gael} % Your first name
\familyname{Sola} % Your last name

% All information in this block is optional, comment out any lines you don't need
\title{Forest monitoring expert}
\address{19/9 Pho Gia Thuong}{Long Bien, Ha Noi, Viet Nam}
%\phone{+84 9 63 19 85 44 (vi)}
\mobile{+84 9 63 19 85 44 (vi)}
\mobiletwo{+33 6 42 85 08 31 (fr)}
%\fax{(000) 111 1113}
\email{gaelsola@hotmail.fr}
%\homepage{staff.org.edu/~jsmith}{staff.org.edu/$\sim$jsmith} % The first argument is the url for the clickable link, the second argument is the url displayed in the template - this allows special characters to be displayed such as the tilde in this example
\extrainfo{skype: gael.sola}
%\photo[70pt][0.4pt]{pictures/photo} % The first bracket is the picture height, the second is the thickness of the frame around the picture (0pt for no frame)
\quote{At the interface between countries technical experts and decision makers to improve forest management and protection}
%\quote{From better data to better decision-making for forest management and protection}

%----------------------------------------------------------------------------------------

\begin{document}

%----------------------------------------------------------------------------------------
%	CURRICULUM VITAE
%----------------------------------------------------------------------------------------

\makecvtitle % Print the CV title


%----------------------------------------------------------------------------------------
%	WORK EXPERIENCE SECTION
%----------------------------------------------------------------------------------------

%\vspace{-6pt}
\section{Experience}

\subsection{FAO - Food and Agriculture Organization of the United Nations}

%\cventry{2014--present}{Forestry and REDD+ technical expert}{UN-REDD Programme, Food and Agriculture Organization of the United Nations}{}{}{
\cventry{2014--present}{Forestry and REDD+ technical expert}{UN-REDD Programme}{Viet Nam and South-Southeast Asia}{}{
\begin{itemize}
\item \textbf{Viet Nam}:
	\begin{itemize}
	\item Provide technical support to the forest department and its technical bodies, in collaboration with international organisations (UNDP, JICA) to develop and improve forest monitoring systems,
	\item Pilot \textbf{updated methods} for (1) forest cover and cover change maps, (2) national forest inventory field manual and quality assessment/quality control,
	\item Organize \textbf{participatory workshops} on monitoring systems for forest activities in six provinces,
	\end{itemize}
\item \textbf{South and Southeast Asia}: 
  \begin{itemize}
  \item Provide \textbf{training} and technical backstopping on forest monitoring to government bodies and research institutions (Bangladesh, Cambodia, the Philippines, Thailand).
  \item Support the design, implementation and update of \textbf{National Forest Inventories} (Bangladesh, Cambodia, Thailand).
  \end{itemize}
\end{itemize}
}

%------------------------------------------------

%\cventry{2012--2014}{Forest inventory and biomass assessment expert}{UN-REDD programme, Food and Agriculture Organization of the United Nations}{FAO HQ, Italy}{}{
\cventry{2012--2014}{Forest inventory and biomass assessment expert}{UN-REDD programme}{FAO HQ, Italy}{}{
\begin{itemize}
\item \textbf{Conduct training}, \textbf{develop guidelines} and training materials for field work, data exploration and data analysis on developing tree biomass allometric equations (Vietnam, Cambodia, Zambia, Tanzania, Congo, DRC).
\end{itemize}
}

%-----------------------------------------%

\subsection{Research Institutes / NGOs}

\cventry{2011--2012}{Research assistant}{Centre de coopération internationale en recherche agronomique pour le développement (CIRAD)}{Republic of the Congo}{}{Installation and monitoring of experimental sites, contribution to data analysis and scientific production.}

\cventry{2009}{Technical assistant}{Groupe energies renouvelables, 
environnement et solidarités \textbf{(GERES)}}{Cambodia}{6 months}{
Wood supply and demand analysis on Phnom Chumriey area.}

\cventry{2008}{Research assistant}{(CIRAD)}{Republic of the Congo}{6 months}{
Study of the nutrient losses by erosion and run off intensity on eucalyptus plantations.}


%----------------------------------------------------------------------------------------
%	EDUCATION SECTION
%----------------------------------------------------------------------------------------

\section{Education}

\cventry{2006--2010}{Master of forest science}{AgroParisTech-ENGREF}{France}{}{
\begin{itemize}%
	\item \textbf{Sustainable} management of \textbf{tropical} and temperate \textbf{forests}.
	\item Forest and \textbf{climate change} mitigation (MDP, REDD).
	\item \textbf{International policy}, \textbf{economy} and \textbf{sociology} applied to \textbf{natural resources} management.
\end{itemize}
}


\subsection{Masters Thesis}

\cvitem{Title}{Formal and informal wood supply chains in two forest areas of Madagascar East Coast (CIFOR-HELVETAS)}
\cvitem{Supervisors}{Professor Jean-Laurent Pfund}
\cvitem{Description}{A deep dive into the harsh reality of the illegal wood commodity chains in and around forest protected areas, revealing the importance of timber logging for local communities around protected areas and quantifying the distribution of revenues from the forest to the main markets.}


%\subsection{Online courses}

%\cvitem{HarvardX}{Statistics and R}
%\cvitem{Microsoft}{Introduction to python: Fundamentals}
%\cvitem{Launchschool}{Introduction to SQL}


%----------------------------------------------------------------------------------------
%	LANGUAGES SECTION
%----------------------------------------------------------------------------------------

\section{Languages}

\begin{tabular}{ m{2cm} m{2.5cm} m{0.5cm} 
                 m{2cm} m{2.5cm} m{0.5cm}
                 m{2cm} m{2.5cm} m{0.5cm} }
French     & \faStar \faStar  \faStar  \faStar  \faStar  & & 
English    & \faStar \faStar  \faStar  \faStar  \faStarO & & 
Vietnamese & \faStar \faStarHalfO \faStarO \faStarO \faStarO & \\

\end{tabular}

%\cvitemwithcomment{English}{Mothertongue}{}
%\cvitemwithcomment{Spanish}{Intermediate}{Conversationally fluent}
%\cvitemwithcomment{Dutch}{Basic}{Basic words and phrases only}


%----------------------------------------------------------------------------------------
%	COMPUTER SKILLS SECTION
%----------------------------------------------------------------------------------------

\section{Computer skills}

\cvitem{Basic}{\textsc{SQL}, \textsc{html}, SAS}
\cvitem{Intermediate}{Geographic Information Systmes (QGIS, ArcGIS), \textsc{python}, \LaTeX, OpenOffice, Linux, Microsoft Windows}
\cvitem{Advanced}{\textsc{R software} for data analysis, statistics, modeling, geospatial analysis and reporting (reports, dashboards), MS Office}

%----------------------------------------------------------------------------------------
%	INTERESTS SECTION
%----------------------------------------------------------------------------------------

\section{Interests}

\renewcommand{\listitemsymbol}{-~} % Changes the symbol used for lists

\cvlistdoubleitem{Nature, Hiking, Photography}{Music \& Music festivals}
\cvlistdoubleitem{Cooking}{Sport}


%----------------------------------------------------------------------------------------
% REFERENCES
%----------------------------------------------------------------------------------------

\section{Bibliography}

\nocite{*}
\bibliographystyle{fbs}
\bibliography{publications}                  


%----------------------------------------------------------------------------------------
%	COVER LETTER
%----------------------------------------------------------------------------------------

% To remove the cover letter, comment out this entire block

% \clearpage
% 
% \recipient{HR Department}{Corporation\\123 Pleasant Lane\\12345 City, State} % Letter recipient
% \date{\today} % Letter date
% \opening{Dear Sir or Madam,} % Opening greeting
% \closing{Sincerely yours,} % Closing phrase
% \enclosure[Attached]{curriculum vit\ae{}} % List of enclosed documents
% 
% \makelettertitle % Print letter title
% 
% \lipsum[1-2] % Dummy text
% \lipsum[4] % Dummy text
% 
% \makeletterclosing % Print letter signature
% 
% \newpage


%----------------------------------------------------------------------------------------

\end{document}